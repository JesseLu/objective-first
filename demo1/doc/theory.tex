\documentclass{article}
\usepackage{amsmath}
\usepackage{hyperref}

\newcommand{\be}{\begin{equation}}
\newcommand{\ee}{\end{equation}}
\newcommand{\grad}{\nabla}
\newcommand{\dvg}{\nabla\cdot}
\newcommand{\curl}{\nabla\times}
\newcommand{\eps}{\epsilon}
\newcommand{\inv}{\frac{1}}
\newcommand{\del}{\partial}

\title{Electromagnetic Theory Handbook for Objective-First Optimization}
\author{Jesse Lu, \texttt{jesselu@stanford.edu}}

\begin{document}
\maketitle
\tableofcontents

\section{Maxwell's equations}
According to Eqs. 3.7 and 3.8 in \cite{TH}, Maxwell's time-harmonic equations ($E$, $H$, $J$, and $M \propto e^{-i \omega t}$) are
\be -i \omega H = -\inv{\mu} \curl E - \inv{\mu} M \ee
\be -i \omega E = \inv{\eps} \curl H - \inv{\eps} J \ee
where $M$ and $J$ are the magnetic and electric current densities, respectively.

The wave equations are then,
\be \curl \inv{\mu} \curl E - \omega^2 \eps E = i \omega J - 
    \curl \inv{\mu} M \ee
    and
\be \curl \inv{\eps} \curl H - \omega^2 \mu H = i \omega M + 
    \curl \inv{\eps} J. \ee

\section{Maxwell's equations for a waveguide.}
Assume that we have a uniform waveguide (not periodic, since periodic waveguides such as photonic crystal waveguides must be dealt with using Bloch's theorem). We want to find solutions of the form,
\be E(x,y,z,t) = E(x,y)e^{i \beta z - i \omega t}, \ee
where $\beta$ is the wave-vector in the direction of propagation ($z$). 

The solution for a two-dimensional waveguide of non-magnetic material ($\mu = \mu_0$ everywhere) is messy\cite{WG}, but the end result is 
\be \left( \grad\inv{\eps_z}\dvg\eps - \curl\curl 
    + \mu_0\omega^2\eps - \beta^2 \right) E_t = 0, \ee
where the transverse $E$-field components are $E_t = \hat{x}E_x + \hat{y}E_y$.
We can then back-out the longitudinal component $E_z$ using
\be \dvg \eps E = 0, \ee
resulting in 
\be E_z = \frac{i}{\beta \eps_z} \dvg \eps E_t. \ee
And then to find the $H$-field components we use
\be H = \inv{i \omega} \curl E, \ee
where $\del z \rightarrow i \beta$.

If there is no variation in $y$ (slab or one-dimensional waveguide), 
then we obtain 
\be \left(\frac{\del}{\del x}\inv{\eps_z}\frac{\del}{\del x}\eps_x
    + \mu_0\omega^2\eps_x - \beta^2 \right) E_x = 0. \ee


\section{Perfectly matched layers}
The upshot of ref.~\cite{SJ} is that a PML can be implemented by simply substituting partial derivatives in the following manner,
\be \frac{\delta}{\delta x} \rightarrow
    \inv{1 + i\frac{\sigma_x(x)}{\omega}} \frac{\delta}{\delta x}, \ee
where $\sigma_x(x) > 0$ in the PML and $\sigma_x = 0$ outside of it.

Further considerations include complex $\sigma$, $\text{Im }\sigma < 0$, to attenuate evanescent waves. Quadratic or cubic growth of $\sigma$ to reduce numerical reflections arising from discretization error. 

Generally, a half-wavelength thick PML layer is sufficient for acceptable attentuation.

\begin{thebibliography}{99}
\bibitem{TH} Allen Taflove, Susan C. Hagness, \emph{Computational Electrodynamics, Third Edition} (Artech House, 2005). 
\bibitem{WG} Jesse Lu, \emph{2.5D Waveguide Equations.pdf} and \emph{2.5D Waveguide Equations (simplified).pdf}, \url{https://github.com/JesseLu/misc/tree/master/scribbling}.
\bibitem{SJ} Steven G. Johnson, \emph{Notes on Perfectly Matched Layers (PMLs)} (2007).
\end{thebibliography}
\end{document}
