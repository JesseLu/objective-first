\documentclass{article}
\usepackage{amsmath}

\newcommand{\be}{\begin{equation}}
\newcommand{\ee}{\end{equation}}
\newcommand{\curl}{\nabla\times}
\newcommand{\eps}{\epsilon}
\newcommand{\inv}{\frac{1}}

\title{Electromagnetic Theory Handbook for Objective-First Optimization}
\author{Jesse Lu, \texttt{jesselu@stanford.edu}}

\begin{document}
\maketitle
\tableofcontents

\section{Maxwell's equations}
According to Eqs. 3.7 and 3.8 in \cite{TH}, Maxwell's time-harmonic equations are
\be -i \omega H = -\inv{\mu} \curl E - \inv{\mu} M \ee
\be -i \omega E = \inv{\epsilon} \curl H - \inv{\eps} J \ee
where $M$ and $J$ are the magnetic and electric current densities, respectively.

The wave equations are then,
\be \curl \inv{\mu} \curl E - \omega^2 \eps E = i \omega J - 
    \curl \inv{\mu} M \ee
\be \curl \inv{\eps} \curl H - \omega^2 \mu H = i \omega M + 
    \curl \inv{\eps} J \ee

\section{Perfectly matched layers}
The upshot of ref.~\cite{SJ} is that a PML can be implemented by simply substituting partial derivatives in the following manner,
\be \frac{\delta}{\delta x} \rightarrow
    \inv{1 + i\frac{\sigma_x(x)}{\omega}} \frac{\delta}{\delta x}, \ee
where $\sigma_x(x) > 0$ in the PML and $\sigma_x = 0$ outside of it.

Further considerations include complex $\sigma$, $\text{Im }\sigma < 0$, to attenuate evanescent waves. Quadratic or cubic growth of $\sigma$ to reduce numerical reflections arising from discretization error. 

Generally, a half-wavelength thick PML layer is sufficient for acceptable attentuation.

\begin{thebibliography}{99}
\bibitem{TH} Allen Taflove, Susan C. Hagness, \emph{Computational Electrodynamics, Third Edition} (Artech House, 2005). 
\bibitem{SJ} Steven G. Johnson, \emph{Notes on Perfectly Matched Layers (PMLs)} (2007).
\end{thebibliography}
\end{document}
