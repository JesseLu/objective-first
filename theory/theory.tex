\documentclass{article}
\usepackage{amsmath}
\usepackage{hyperref}
\usepackage{graphicx}

\newcommand{\be}{\begin{equation}}
\newcommand{\ee}{\end{equation}}
\newcommand{\grad}{\nabla}
\newcommand{\dvg}{\nabla\cdot}
\newcommand{\curl}{\nabla\times}
\newcommand{\eps}{\epsilon}
\newcommand{\inv}{\frac{1}}
\newcommand{\del}{\partial}
\newcommand{\der}[1]{\frac{\del}{\del #1}}
\newcommand{\BI}{\begin{itemize}}
\newcommand{\I}{\item}
\newcommand{\EI}{\end{itemize}}
\newcommand{\MIN}[1]{\underset{#1}{\text{minimize}}\quad}
\newcommand{\ST}{\text{subject to}\quad}
\newcommand{\diag}{\text{diag}}


\title{Theory for Boundary-Value Objective-First Optimization}
\author{Jesse Lu, \texttt{jesselu@stanford.edu}}

\begin{document}
\maketitle
\tableofcontents

\section{Improving the field}
We begin by writing down a generic (sourceless) physics problem, 
\be A(p)x = 0. \label{orig_eq} \ee
Here, $x$ is the field variable, $p$ is the structure variable, and $A(p)$ represents the physics of the problem.

Now, suppose that we want to fix the value of the field at certain grid-points. 
We denote these fixed points as \emph{boundary values} of the problem.
Specifically, we break up $x$ into
\be x = S_1 x_1 + S_0 x_0, \ee
where $S_1$ and $S_0$ are selection matrices for the varying and fixed elements of x respectively. 

We can now attempt to solve eq.~\ref{orig_eq} by finding $x_1$ in the following manner. Let
\be A(p)S_1 x_1 = -S_0 x_0, \ee
or
\be \hat{A}(p) x_1 = b, \label{field_eq} \ee
where $\hat{A}(p) = A(p) S_1$ and $b = -S_0 x_0$.

Note that there often will not be a valid $x_1$ to satisfy eq.~\cite{field_eq} since $\hat{A}(p)$ will in general be skinny and full-rank. 
Instead, we can minimize the \emph{physics residual}, defined as
\be \| A(p)x \|^2 = \| \hat{A}(p) x_1 - b \|^2. \ee

We call solving the problem
\be \MIN{x_1} \| \hat{A}(p)x_1 - b \|^2\ee
improving the field.
Note that an equivalent problem definition is
\begin{align}
    \MIN{x}& \| A(p)x \|^2 \\
    \ST& S_0^T x = x_0.
\end{align}


\section{Improving the structure}
We now consider the structure improvement problem, that is,
\be \MIN{p} \| A(p)x \|^2. \label{struct_eq}\ee

Whereas the field improvement problem can always be solved exactly, this is not the case for the structure improvement problem. 
Even when $A(p)$ is linear with respect to $p$, there are often restrictions on $p$ which make finding a solution very difficult. 
For example, a common restriction is that $p$ be binary, that is, $p \in {p_0, p_1}$. 
In this case, the problem is generally NP-hard.
For these reasons, we often use a simple gradient-descent method to arrive at an approximate solution of eq.~\ref{struct_eq}.

If $A(p) = A_1 \diag(A_2 p) A_3$ then,
\begin{align}
    A(p)x &= A_1 \diag(A_2 p) A_3 x \\
        &= A_1 \diag(A_3 x) A_2 p \\
        &= B(x) p. 
\end{align}
where $B(x) = A_1 \diag(A_3 x) A_2$.

The gradient of the physics residual with respect to $p$ can then be computed using
\be \der{p} \inv{2} \|A(p)x\|^2 = \der{p} \inv{2} \|B(x)p\|^2 = B(x)^*B(x)p. \ee

\section{Gradients}
A first-order approximation to a function, $f(x): \mathcal{C}^{n} \to \mathcal{R}$ can be formulated as
\be f(x) \approx f(x_0) + \mathcal{R}\{g(x_0)^\ast (x - x_0)\}, \ee
where $g(x_0)$ is the gradient of $f$ at $x_0$.

To test that $g(x)$ does indeed give the correct value for the gradient, one can produce a set random of vectors $v_n$ and find the error, 
\be \frac{\| (f(x_0+v_n) - f(x_0)) - \mathcal{R}\{g(x_0)^\ast (v_n)\} \|^2}
	{\| v_n \|^2} \ee
for $\|v_n\|^2 \ll 1 $.
	

\begin{thebibliography}{99}
\end{thebibliography}
\end{document}
