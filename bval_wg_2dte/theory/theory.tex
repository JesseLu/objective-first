\documentclass{article}
\usepackage{amsmath}
\usepackage{hyperref}
\usepackage{graphicx}

\newcommand{\be}{\begin{equation}}
\newcommand{\ee}{\end{equation}}
% \newcommand{\bz}{\begin{equation}}
% \newcommand{\ee}{\end{equation}}
\newcommand{\grad}{\nabla}
\newcommand{\dvg}{\nabla\cdot}
\newcommand{\curl}{\nabla\times}
\newcommand{\eps}{\epsilon}
\newcommand{\inv}{\frac{1}}
\newcommand{\del}{\partial}
\newcommand{\dt}{\frac{\del}{\del t}}
\newcommand{\BI}{\begin{itemize}}
\newcommand{\I}{\item}
\newcommand{\EI}{\end{itemize}}
\newcommand{\MIN}[1]{\underset{#1}{\text{minimize}}\quad}
\newcommand{\ST}{\text{subject to}\quad}


\title{Theory for Boundary-Value Objective-First Optimization}
\author{Jesse Lu, \texttt{jesselu@stanford.edu}}

\begin{document}
\maketitle
\tableofcontents

\section{Setting up the boundary-value problem}
We begin by writing down a generic (sourceless) physics problem, 
\be A(p)x = 0. \label{orig_eq} \ee
Here, $x$ is the field variable, $p$ is the structure variable, and $A(p)$ represents the physics of the problem.

Now, suppose that we want to fix the value of the field at certain grid-points. 
We denote these fixed points as \emph{boundary values} of the problem.
Specifically, we break up $x$ into
\be x = S_1 x_1 + S_0 x_0, \ee
where $S_1$ and $S_0$ are selection matrices for the varying and fixed elements of x respectively. 

We can now attempt to solve eq.~\ref{orig_eq} by finding $x_1$ in the following manner. Let
\be A(p)S_1 x_1 = -S_0 x_0, \ee
or
\be \hat{A}(p) x_1 = b, \label{field_eq} \ee
where $\hat{A}(p) = A(p) S_1$ and $b = -S_0 x_0$.

Note that there often will not be a valid $x_1$ to satisfy eq.~\cite{field_eq} since $\hat{A}(p)$ will in general be skinny and full-rank. 
Instead, we can minimize the \emph{physics residual}, defined as
\be \| A(p)x \|^2 = \| \hat{A}(p) x_1 - b \|^2. \ee

We call solving the problem
\be \MIN{x_1} \| \hat{A}(p)x_1 - b \|^2\ee
improving the field.
Note that an equivalent problem definition is
\begin{align} 
    \MIN{x}& \| A(p)x \|^2 \\
    \ST& S_0^T x = x_0.
\end{align}


\begin{thebibliography}{99}
\end{thebibliography}
\end{document}
